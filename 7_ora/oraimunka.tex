\documentclass{article}
\usepackage[magyar]{babel}
\usepackage{t1enc}
\usepackage{lipsum}
\usepackage{hulipsum}
\usepackage{amsmath}
\usepackage{amsfonts}
\usepackage{amssymb}
\usepackage{mathtools}
\usepackage{xcolor}
\usepackage{colortbl}
\usepackage{array}
\usepackage{ifthen}
\usepackage{pgffor}
\usepackage{multicol}
\usepackage{intcalc}


\DeclareMathOperator*{\argmax}{arg\,max}
\DeclarePairedDelimiter{\ceil}{\lceil}{\rceil}
\DeclarePairedDelimiter{\kocka}{[}{]}
\newcommand{\VE}[1]{\mathbb{E}\kocka*{#1}}
\DeclarePairedDelimiterX{\vonal}[2]{[}{]}{#1 \;\delimsize\vert\; #2}
\newcommand{\FVE}[2]{\mathbb{E}\vonal*{#1}{#2}}

\newenvironment{csikos}%
{\vspace{1ex}\hrule\vspace{1ex}}%
{\vspace{1ex}\hrule\vspace{1ex}}

\newenvironment{csikos*}[1][Kulcsgondolatok]{\begin{center}\begin{minipage}{0,8\textwidth}%
\vspace{1ex}\hrule\vspace{1ex}\begin{center}\Large\textbf{#1}\end{center}}{\vspace{1ex}\hrule\vspace{1ex}%
\end{minipage}\end{center}}


\begin{document}
\begin{huge}
1. feladat
\end{huge}
\\
\[
X^* := \argmax_{x\in[0, 1]} x \log_2(x)
\]

\[
\ceil{x}, \ceil*{\frac{5}{3}}
\]

\[\VE{\sum_{i=1}^{N}{X_i}} = \VE{\FVE{\sum_{i=1}^{N}{x_i}}{N}}\]
\clearpage
\begin{huge}
2. feladat
\end{huge}
\\
\begin{csikos}
\lipsum[3]
\end{csikos}

\begin{csikos*}
\lipsum[4]
\end{csikos*}

\clearpage
\begin{huge}
3. feladat
\end{huge}
\\

\begin{csikos*}
\newcommand{\kgitem}{\par\makebox[1 cm]{\stepcounter{szamlalo} \theszamlalo}}
\newcounter{szamlalo}
\newcounter{osszamlalo}
\counterwithin{szamlalo}{osszamlalo}
\kgitem
\section{Elso}
\stepcounter{osszamlalo}
\kgitem
csiga
biga
toldki
\kgitem
\kgitem
\kgitem
\kgitem
\section{Masodik}
\stepcounter{osszamlalo}
\kgitem
\lipsum[2]
\kgitem
\kgitem csiga
\kgitem biga
A számláló értéke: \theszamlalo
\end{csikos*}

\clearpage



\newcolumntype{h}[2]{>{%
\ifthenelse{%
\isodd{\value{rownum}}}%
{%
\cellcolor{#1}\color{#2}}%
{\cellcolor{#2}\color{#1}%
}%
}c}%

\begin{huge}
4. feladat
\end{huge}

\begin{table}
\rowcolors[]{2}{blue!20}{yellow!20}
\begin{tabular}{h{white}{black} h{white}{black} h{white}{black}}
a & b & c \\
a & b & c \\
a & b & c \\
a & b & c \\
a & b & c \\
a & b & c \\
\end{tabular}
\end{table}

\clearpage

\begin{huge}
5. feladat
\end{huge}

\newcounter{y}
\newcounter{x}

\begin{equation}
\foreach \x/\y in {1/4, 2/7, 4/3, 3/2, 5/2, 6/8} %
{ \frac{\x}{\y}, }
\end{equation}

\clearpage

\begin{huge}
6. feladat
\end{huge}

\newcounter{a}
\newcounter{b}
\newcounter{c}

\begin{multicols}{4}
\whiledo{\value{b}<60}{\stepcounter{b}\stepcounter{a}\stepcounter{c}\arabic{b}~%
\ifthenelse{\equal{\intcalcMod{\value{c}}{3}}{0}}{cikk}{}%
\ifthenelse{\equal{\intcalcMod{\value{a}}{5}}{0}}{cakk}{}%
\\}
\end{multicols}

\end{document}