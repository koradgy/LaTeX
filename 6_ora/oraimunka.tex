\documentclass{article}
\usepackage{amsmath}
\usepackage{mathtools}
\usepackage{amsfonts}
\usepackage{amssymb}
\usepackage{xcolor}


\begin{document}

a) Az $\frac{1}{n^2}$ sorösszege:
\[
\sum_{n=1}^\infty \frac{1}{n^2}=\frac{\pi^2}{6}.
\]
b) Az n! (n faktoriális) a számok szorzata 1-től n-ig, azaz
\[
n! := \prod_{k=1}^n k=1 \cdot 2 \cdot 3 \cdot \cdots \cdot n.
\]
Konvenció szerint 0!=1
\\
c) Legyen 0 $\leq$ k $\leq$ n a binomiális együttható
\[
\binom{n}{k} = \frac{n!}{k! \cdot (n - k)!}
\]
ahol a faktoriálist (\textcolor{red}{1}) szerint definiáljuk. 
\\
d) Az előjel- azaz szignum függvényt a következőképpen definiáljuk:
\[
sgn(x) := 
\begin{cases}
1 \text{ ha x $>$ 0},\\
0 \text{ ha x = 0} ,\\
-1 \text{ ha x $<$ 0},
\end{cases}
\]

\clearpage
\begin{huge}
2: Deretmináns
\end{huge}

a) Legyen
\[
[n] := \{1, 2, 3, \cdots ,n \}
\]
a természetes számok halmaza 1-től n-ig.\\
b) Egy n-edredű \emph{permutáció} $\sigma$ bijekció [n]-ből [n]-be. Az n-edrendű permutációk halmazát, az ún. szimmetrikus csoportot, Sn-nel jelöljük.\\
c) Egy $\sigma$ $\in$ Sn permutációban inverziónak nevezünk egy (i,j) párt ha i $<$ j, de $\sigma$i $>$ $\sigma$j.\\
d)Egy  $\sigma$ $\in$ Sn permutáció paritásának az inverziók számát nevezzük:
\[
\mathcal{I}(\sigma) := \vert \{(i,j) | i,j \in [n], i<j, \sigma_{i} > \sigma_{j}\} \vert
\]
\\
e) Legyen A $\in$ $\mathbb{R}^{n \times n}$ , egy n $\times$ n-es (négyzetes) valós mátrix:

\[
A=
\begin{pmatrix}
a_{11} & a_{12} & \dots & a_{1n}\\
a_{21} & a_{22} & \dots & a_{2n}\\
\vdots & \vdots & \ddots & \vdots\\
a_{n1} & a_{n2} & \dots & a_{nn}
\end{pmatrix}
\]
\\
A mátrix determinánsát a következőképpen definiáljuk:

\[
A=
\begin{vmatrix}
a_{11} & a_{12} & \dots & a_{1n}\\
a_{21} & a_{22} & \dots & a_{2n}\\
\vdots & \vdots & \ddots & \vdots\\
a_{n1} & a_{n2} & \dots & a_{nn}
\end{vmatrix}
 := \sum_{\sigma \in S_{n}} (-1)^{\mathcal{I}(\sigma)} \prod_{i=1}^n a_{i_{\sigma_{i}}}
\]
\clearpage

\begin{huge}
3. Logikai azonosság
\end{huge}

Tekintsük az L = \{0 , 1\} halmazt, és rajta a következő, igazságtáblával definiált műveleteket:
\\
\\
\begin{table}[!htb]

\begin{tabular}{c||c}
$\mathcal{X}$ & $\overline{X}$ \\ \hline
0 & 1 \\
1 & 0

\end{tabular}

\begin{tabular}{c|c||c|c|c}
$\mathcal{X}$ & $\mathcal{Y}$ & x$\lor$y & x$\land$y & x $\to$ y\\ \hline
0 & 0 & 0 & 0 & 1 \\
0 & 1 & 1 & 0 & 1 \\
1 & 0 & 1 & 0 & 0 \\
1 & 1 & 1 & 1 & 1
\end{tabular}
\end{table}

Legyenek a, b, c, d $\in$ L. Belátjuk a következő azonosságot.

\[
(a \land b \land c) \to d = a \to (b \to (c \to d)).
\]

A következő azonosságokat bizonyítás nélkül használjuk.


 

\end{document}


