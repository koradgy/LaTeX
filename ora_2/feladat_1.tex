\documentclass{book}
\usepackage[magyar]{babel}
\usepackage{hulipsum}
\usepackage[inner=3cm,outer=5cm]{geometry}
\geometry{bindingoffset=1cm}
\geometry{marginparwidth=3cm,marginparsep=0.5cm}

\begin{document}
\title{Bevezetés}
\author{Kórád György}
\date{\today}



\maketitle

\renewcommand{\thefootnote}{\fnsymbol{footnote}}
\setcounter{tocdepth}{5}
\tableofcontents{\pagenumbering{roman}}

\clearpage

%\begin{abstract}
%\hulipsum[1]
%\hulipsum[1]
%\footnote{Lábjegyzet szövege}
%
%\end{abstract}
\pagenumbering{arabic}
\section{Első Section}\footnote{Section footnote}
\hulipsum[1]
\subsection{Sub 1}
\hulipsum
\clearpage
\subsection{Sub 2}
\hulipsum
\clearpage
\setcounter{secnumdepth}{5}
\section[Rövid név]{Második Section}
\hulipsum[1]
\part{}
\hulipsum[1]
\section{Section}
\hulipsum[1]
\subsection{Subsection}
\hulipsum[1]
\subsubsection{Subsubsection}
\hulipsum[1]
\paragraph{Paragraph}
\hulipsum[1]
\subparagraph{Subparagraph}
\hulipsum[1]
\clearpage
\appendix
\section{Függelék}
\subsection{}
\hulipsum[1]
\subsection{}
\hulipsum[1]
\marginpar{margó szöveg pls}
\section{Függelék 2}
\subsection{}
\hulipsum[1]
\subsection{}
\hulipsum[1]
\clearpage
\quote
\hulipsum[1-2]
\clearpage
\quotation
\hulipsum[1-2]
\clearpage
\begin{verse}
\hulipsum[1-2]
\end{verse}
\end{document}