\documentclass[twocolumn]{report}
\usepackage{geometry}
\usepackage{fancyhdr}
\fancyhead[R]{\thepage}
\fancyhead[L]{\nouppercase{\leftmark}}
\fancyfoot[C]{xd csoport}
\fancypagestyle{plain}{%
\fancyhf{}%
\fancyhead[R]{\thepage}}
\usepackage{t1enc}
\frenchspacing
\usepackage{hulipsum}

\usepackage{array}
\usepackage[table]{xcolor}
\usepackage{multirow}

\usepackage{amsmath}
\usepackage{mathtools} 
\usepackage{amsfonts}
\usepackage{amsthm}
\theoremstyle{definition}

\usepackage[magyar]{babel}

\newtheorem{defin}{Definíció}
\newtheorem{tet}{Tétel}

\usepackage{algpseudocode}
\usepackage{algorithm} 
\floatname{algorithm}{Algoritmus}

\begin{document}
\pagestyle{fancy}

\title{Zárthelyi Dolgozat \\\Large xy csoport}
\author{György ZF440N}
\date{\today}
\maketitle
\tableofcontents

\chapter{Első feladat}
\section{Első szakasz}
\hulipsum[1-12]
\section{Második szakasz}
\hulipsum[13-24]

\chapter{Táblázat}
\begin{table*}
\caption{Egy vállalat kimutatása}
\begin{center}
\begin{tabular}{c|c|>{\columncolor{green!30}}r>{\columncolor{red!30}}r>{\columncolor{yellow!30}}r}
év & ágazat & bevétel & kiadás & nettó bevétel \\ \hline
\multirow{4}{*}{2020} & marketing & xx & xx & xx \\
 & humán erőforrás & \multicolumn{3}{c}{N/A} \\
 & logisztika & xx & xx & xx \\
 & gyártás & xx & xx & xx \\ \hline
\multirow{4}{*}{2021} & marketing & \cellcolor{white}N/A & xx & \cellcolor{white}N/A \\
 & humán erőforrás & xx & xx & xx \\
 & logisztika & xx & xx & xx \\
 & gyártás & xx & \multicolumn{2}{c}{N/A}
\end{tabular}
\end{center}
\end{table*}

\hulipsum

\chapter{Matematika}
\begin{defin}[Mátrix szorzás]
Legyenek $m,\textcolor{red}n,r \in \mathbb{Z}^+$, $A \in \mathbb{R}^{m \times \textcolor{red}n}$, és $B \in \mathbb{R}^{\textcolor{red}n \times r}$. Bontsuk fel $A$-t sorvektoraival, $B$-t pedig oszlopvektoraival.
\begin{equation}
A = \begin{pmatrix}
a_1^T\\
\cellcolor{red!30}\hphantom{a_2^T} a_2^T \hphantom{a_2^T}\\
\vdots \\
a_m^T\\
\end{pmatrix}
\quad
B = \begin{pmatrix}
 & \cellcolor{blue!30} \\
b_1 & \cellcolor{blue!30} b_2 & \cdots & b_r  \\
 & \cellcolor{blue!30}
\end{pmatrix}
\end{equation}
Mind $a_i$, mind $b_j \in \mathbb{R}^{\textcolor{red}{n}}$, így az $a_i^T b_j$ skalár szorzat létezik. 
Az $A \cdot B$ mátrix szorzatot a következőképpen definiáljuk:

\begin{equation}
A \cdot B := \begin{pmatrix}
a_1^T b_1 & \cellcolor{blue!15}a_1^T b_2 & \cdots & a_1^T b_r \\
\rowcolor{red!15} a_2^T b_1 & \cellcolor{purple!30} a_2^T b_2 & \cdots & a_2^T b_r\\
\vdots & \cellcolor{blue!15} \vdots & \ddots & \vdots \\
a_m^T b_1 & \cellcolor{blue!15}a_m^T b_2 & \cdots & a_m^T b_r
\end{pmatrix}
\end{equation}
\end{defin}

\begin{tet}[Mátrix szorzás tulajdonságai]
A mátrix szorzás asszociatív, de nem kommutatív művelet.
\end{tet}
\begin{tet}[Mátrix szorzás determinánsa]
Ha $A,b \in \mathbb{R}^{n \times n}$ négyzetes mátrixok, akkor
\[det(A \cdot B) = det(A) \cdot det(B).\]
\end{tet}

\chapter{Pszeudokód}

\hulipsum[1]
\begin{algorithm}
\caption{Lineáris keresés}
\begin{algorithmic}[3]
\Procedure{linesearch}{A,ertek,@index}
\Require A tömb érték a keresett érték
\Ensure index (az első) A-beli index, hogy $A_i$ = érték, vagy érvénytelen (0)
\State i $\gets$ 1
\While{$A_i \neq$ ertek \textbf{and} i < Hossz[A]}
\State \Call{inc}{i}
\EndWhile
\If{i < Hossz[A]}
\State index $\gets$ i
\Else
\State index $\gets$ 0
\EndIf
\State \Return index
\EndProcedure
\end{algorithmic}
\end{algorithm}
\hulipsum
\end{document}