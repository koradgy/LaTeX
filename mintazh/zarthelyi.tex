\documentclass[twocolumn]{report}
\usepackage{geometry}
\usepackage{fancyhdr}
\fancyhead[R]{\thepage}
\fancyhead[L]{\nouppercase{\leftmark}}
\fancyfoot[C]{xd csoport}
\fancypagestyle{plain}{%
\fancyhf{}%
\fancyhead[R]{\thepage}}
\usepackage{t1enc}
\frenchspacing
\usepackage{hulipsum}

\usepackage{array}
\usepackage[table]{xcolor}
\usepackage{multirow}

\usepackage{amsmath}
\usepackage{mathtools} 
\usepackage{amsfonts}
\usepackage{amsthm}
\theoremstyle{definition}

\usepackage[magyar]{babel}

\newtheorem{defin}{Definíció}
\newtheorem{tet}{Tétel}

\begin{document}
\pagestyle{fancy}

\title{Zárthelyi Dolgozat \\\Large xy csoport}
\author{György ZF440N}
\date{\today}
\maketitle
\tableofcontents

\chapter{Első feladat}
\section{Első szakasz}
\hulipsum[1-12]
\section{Második szakasz}
\hulipsum[13-24]

\chapter{Táblázat}
\begin{table*}
\caption{Egy vállalat kimutatása}
\begin{center}
\begin{tabular}{c|c|>{\columncolor{green!30}}r>{\columncolor{red!30}}r>{\columncolor{yellow!30}}r}
év & ágazat & bevétel & kiadás & nettó bevétel \\ \hline
\multirow{4}{*}{2020} & marketing & xx & xx & xx \\
 & humán erőforrás & \multicolumn{3}{c}{N/A} \\
 & logisztika & xx & xx & xx \\
 & gyártás & xx & xx & xx \\ \hline
\multirow{4}{*}{2021} & marketing & \cellcolor{white}N/A & xx & \cellcolor{white}N/A \\
 & humán erőforrás & xx & xx & xx \\
 & logisztika & xx & xx & xx \\
 & gyártás & xx & \multicolumn{2}{c}{N/A}
\end{tabular}
\end{center}
\end{table*}

\hulipsum

\chapter{Matematika}


\end{document}