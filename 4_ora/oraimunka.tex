\documentclass{article}
\usepackage{amsthm}
\usepackage{hulipsum}
\theoremstyle{plain}\newtheorem{tet}{Tétel}
\newtheorem{defin}[tet]{Definíció}
\theoremstyle{plain}\newtheorem{lem}[tet]{Lemma}
\newtheorem{fel}{Feladat}[section]
\newtheorem*{megj}{Megjegyzés}
\usepackage{float}
\newfloat{forraskod}{hbt}{lop}
\usepackage{listings}
\usepackage{algpseudocode}


\begin{document}
\begin{tet} ez egy tétel \end{tet}
\begin{tet}[Gyros pitában] ez egy másik tétel \end{tet}
\begin{defin}Nagyon definíció \end{defin}
\begin{defin}[Úristen very big]Nagyon nagyon definíció \end{defin}
\begin{lem}
aztaz eget
\end{lem}

\begin{fel}
szöveg
\section{Feladat}
szöveg
\section{Feladat}
szöveg
\section{Feladat}
szöveg

\end{fel}

\begin{megj}
Tehát megjegyzés
\end{megj}
\clearpage
\begin{huge}
Második feladat
\end{huge}


\verb\LaTeX \LaTeX \LaTeX

\begin{verbatim}
\begin{megj}
Tehát megjegyzés
\end{megj}
\end{verbatim}
\verb|\texttt{verbatim}| szöveg

\begin{verbatim}
\floatname{forraskod}
\end{verbatim}

\clearpage
\begin{huge}
Harmadik feladat
\end{huge}


\lstinputlisting[language=java,tabsize=7,numbers=left,stepnumber=4,frame=shadowbox]{vl.txt}
\clearpage
\begin{huge}
Negyedik feladat
\end{huge}

\begin{algorithmic}
\Procedure{Quickshort}{@A,a,b}
\Require Írható tömb
\Require 1 $\leq$ a $\leq$ b $\leq$ Hossz[A] indexek
\Ensure a-b indextartományt rendezzük
\If{a=b}
\\
\Return a
\Else
\State{FELOSZT(@A,a,b,A(a),@q)}
\State{QUICKSORT(@A,a,q)}
\State{QUICKSORT(@A,q+1,b)}
\State return A
\EndIf
\EndProcedure
\end{algorithmic}

\end{document}